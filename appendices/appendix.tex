\chapter{Controller Database}
\label{apdx:controllerdb}


\begin{lstlisting}[language=sql, basicstyle=\footnotesize,
]
CREATE TABLE Zones(
id INTEGER NOT NULL,
name TEXT,
PRIMARY KEY(id)
);

CREATE TABLE Sites(
tp_address TEXT NOT NULL,
name TEXT,
PRIMARY KEY(tp_address)
);
	  
CREATE TABLE Subnets(
net_ip BLOB NOT NULL,
net_mask BLOB NOT NULL,
zoneID INTEGER NOT NULL,
tp_address TEXT NOT NULL,
PRIMARY KEY (net_ip, net_mask),
FOREIGN KEY (zoneID) REFERENCES Zones(id) ON DELETE CASCADE,
FOREIGN KEY (tp_address) REFERENCES Sites(tp_address)
	ON DELETE CASCADE
);
	  
CREATE TABLE Transfers(
src INTEGER NOT NULL,
dest INTEGER NOT NULL,
PRIMARY KEY (src, dest) ON CONFLICT REPLACE,
FOREIGN KEY (src) REFERENCES Zones(id) ON DELETE CASCADE,
FOREIGN KEY (dest) REFERENCES Zones(id) ON DELETE CASCADE	
)
\end{lstlisting}

The controller database consists of 4 tables: \texttt{Zones}, \texttt{Sites}, \texttt{Subnets},
and \texttt{Transfers}. The \texttt{Zones} table contains all network zones known to the
controller, identified by zone IDs. Additionally, a human readable description is attached. The
\texttt{Sites} table holds all known branch sites with the addresses of the corresponding \tps
and a textual description. The \texttt{Subnets} table describes the configured IP subnets
together with their zone membership and the \tp behind which they are located. Finally, the
\texttt{Transfers} table reflects the zone transfer matrix of allowed zone transfers.
