\chapter{Related Work}

The following sections provide a brief overview of other testbed environments for running network and distributed systems experiments. In particular, their approach of organising and managing network nodes and their way of handling users enrolled in experiments is highlighted.

\section{PlanetLab}

\fnurl{PlanetLab}{https://www.planet-lab.org/} is an open platform for computer networking and distributed systems research. It allows the deployment and testing of newly developed network protocols, distributed algorithms, peer-to-peer software and \fnote{CDNs}{Content Delivery Networks}. \cite{MA} For conducting tests, each project has access to a set of globally distributed virtual machines. These sets, called slices in PlanetLab terminology, run on physical nodes provided and maintained by users of PlanetLab, mainly research institutions.

\subsection{PlanetLab Node Management}

For an institution to be granted access to PlanetLab, it has to add its own computational resources to the PlanetLab network. \cite{planetlab_two_nodes} These nodes are remotely managed by PlanetLab operational staff. Local administrators are not given root access and are merely allowed to modify certain parameters, such as outgoing network bandwidth. \cite{planetlab_hosting_req} A Principal Investigator (PI) at each site is responsible for approving accounts and assigning them to slices. Furthermore, the PI locally enforces the PlanetLab Acceptable Use Policy (AUP). After instantiating a slice, users are given \fnote{SSH}{Secure Shell} access with root privileges to nodes in their slice. However, this root access is restricted as it does not allow changes to hardware and network configurations. \cite{planetlab_user_guide} 

Comparing to \lee, there are some fundamental differences:

\begin{itemize}
	\item In \lee, each project joins with its own resources. Neither are there restrictions regarding system configuration, nor are nodes managed by SCIONLab operational staff.
	\item There is no assignment to slices by a PI in SCIONLab. The deployment of nodes in SCIONLab is left entirely to the project team.
	\item Deployed nodes do not necessarily need to be virtualized. There are configurations for running SCION directly on physical machines.
	\item Since there are no PIs in SCIONLab, the intermediate step of creating accounts via PI is omitted. Institutions directly register their accounts via \lcs.
	\item SCIONLab uses a novel approach of pre-authenticated accounts with pre-established connections to get research institutions aboard. This allows invited parties to start using SCIONLab with minimal overhead.
	\item Developed from ground up with simplicity in mind, the \lee offers easy-to-use node management tools.
\end{itemize}

\section{GENI}

Similar to PlanetLab, \fnurl{GENI}{http://www.geni.net/} (Global Environment for Network Innovations) is a testbed environment designed for networking and distributed systems research. Like PlanetLab, GENI uses the notion of slices to describe a set of geographically distributed virtual and physical hosts, instantiated for an experiment. The unique feature of GENI is its "deep programmability" capability which lets users connect compute resources on the link layer and replace above layers with custom protocols. \cite{geni}

In GENI, each project is led by a single individual; the project lead. The project lead is responsible for allocating slices and assigning project members to them. Slices consist of different resource providers available in the GENI network, so called aggregates.

Aggregates are hosted by institutions and managed by local operators. In order to start using GENI, an account must be registered. However, thanks to tight collaboration with many institutions, the institutions account may be used to register for GENI. Additionally, since experiments on GENI often revolve around new services, GENI allows end users, who are not affiliated with the project, to opt in, in order to bring real traffic to experiments. \cite{geni}


\section{Fed4FIRE}

\fnurl{FED4FIRE}{https://www.fed4fire.eu/} strives for creating a large federation of experimentation facilities and network testbeds in Europe. The main idea behind this initiative is to simplify the use of already existing testbed environments, thus making it possible for researchers to collaborate by sharing existing test facilities in the broad field of \fnote{ICT}{Information and Communication Technology}. This also allows researchers to use multiple testbeds for their experiments. \cite{fire_book}

FED4FIRE offers its facilities in two ways: Open Calls are selected projects which receive financial support to be carried out. The other possibility, Open Access, allows every interested party to run their projects, without being funded.

Accounts are registered at a FED4FIRE authority and then used to create new projects, similar to PlanetLab and GENI. The architecture consists of multiple testbed environments, all aggregated under the FED4FIRE initiative. By the nature of this very heterogeneous design, the management of the environment is much more complicated than in SCIONLab. \cite{fire_book}