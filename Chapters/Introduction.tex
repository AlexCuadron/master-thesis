\chapter{Introduction}

 SCION \cite{scion_book} is a proposed Future Internet Architecture (FIA) with the goal to improve on today's Internet which suffers from various shortcomings. Many of these shortcomings are related to the IP and BGP protocols which together form the narrow waist in the protocol stack. In particular, the Internet lacks transparency and control over packet routing and the global nature of its protocols makes it scale badly and suffer from outages. Instead of keeping on building on this unstable foundation, SCION tackles these problems with a fundamental re-design of the Internet's core protocols. It aims to provide 
 high availability, control, transparency and secure end-to-end communication in networks. \cite{scion_5yrs}
 
 As such, SCION offers convincing opportunities especially for the industry which can profit greatly from its unique features. 
 The adoption of SCION is not an obstacle as it interfaces nicely with today's architecture. Still, it is necessary that businesses, research institutions and other interested parties are able to test its unique features. For this exact purpose the SCIONLab Experimentation Environment was created. It's a project aiming to provide a testbed environment where users join the SCION network with their own computation and actively contribute to the network, thus allowing for realistic testing scenarios. To facilitate the process of joining SCIONLab and managing nodes, SCIONLab provides an easy-to-use tool. The \lcs offers an intuitive interface enabling interested parties to register and download AS configurations to deploy onto their own hardware in order to become part of the SCION network. Moreover, \lcs is designed to be a global intermediary between the local management services of different ASes, managing connections between one another. Additional information about SCIONLab and \lcs is available in Sections \ref{back:scionlab} and \ref{back_scion_coord} respectively.
  
 This thesis aims to improve on the existing Coordination Service, making it easier and more appealing for end users as well as significantly improving its functionality, security and maintainability. These improvements in particular consist of the following enhancements and extensions: (1) An email verification system used to validate user registrations, (2) an administrator panel that allows to review registrations and activate users, (3) the implementation of a CAPTCHA, in order to protect the service against bot accounts, (4) interfacing with a continuous integration solution for faster testing of \lcs, (5) leveraging an automation engine for easier deployment onto multiple hosts and (6) enhancing the service with numerous improvements addressing speed, cleanliness and security of the system.

The rest of the thesis is structured as follows: Chapter 2 contains an overview of the SCION architecture, briefly describing its components. In Chapter 3, testbed management systems of other network experimentation environments are discussed. Chapter 4 presents design objectives constrained by functional and non-functional requirements. Then, based on the requirements elicited in the previous chapter, we outline the proposed \lcs architecture with its associated components and services in chapter 5. Chapter 6 describes the process of implementing enhancements dictated by the gathered requirements. It's shown how \lcs was transformed to match the proposed architecture. To sum up, Chapter 7 evaluates the enhancements made in chapter 6 and discusses achievements of design objectives from Chapter 4. Finally, in chapter 8 we come to a conclusion and provide an outlook on future work.
 