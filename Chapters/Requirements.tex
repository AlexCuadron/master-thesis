\chapter{Requirements Engineering}
\label{req}


With the introduction of SCIONLab, \lcs now plays a significant role in the \lee. Because of this, a set of new requirements arose, focusing on making the service an intuitive, user friendly tool, while at the same time increasing the robustness of the system. This shift from a mediator to an online account management tool, allowing users to register accounts and download their SCION configurations in order join the network with minimal overhead, demands an extension of initial requirements.

\section{Functional Requirements}
\label{func_req}

To make \lcs meet the functionality outlined above, the following requirements were gathered:

\begin{enumerate}  
	\item \textit{A mechanism to verify users' email addresses:}\\
		This increases the authenticity of registered accounts. This also ensures that users can be contacted by email.
	\item \textit{A mechanism to manually activate users who signed up successfully:}\\
		Not all users should immediately be granted access to \lcs. Administrators need to be able to audit registration requests.
	\item \textit{A mechanism to protect the service against automated account creation:}\\
		This safety measurement protects the service from spam and the creation of fake accounts.
	\item \textit{New functionality is validated using CircleCI's testing environment:}\\
		This increases development productivity and ensures new additions do not break \lcs.
	\item \textit{A fast and easy way to deploy the service onto multiple machines:}\\
		Since \lcs is evolving quickly, it is required that it can be deployed onto the target machine effortlessly.
	\item \textit{A system for sending notifications to users:}\\
		Users should be notified if the status of their ASes change or when new versions of SCION are available.
	\item \textit{Implementation of missing APIs between \cords and the \lmi:}\\
		This ensures flawless operation of \lcs in its role as mediator.
	\item \textit{A visualization of the SCIONLab Experimentation network accessible by users:}\\
		This makes it easy for new users to get an overview of the SCIONLab network and helps them join an ISD.
\end{enumerate}

\section{Non-Functional Requirements}
\label{non-func_req}

Non-Functional requirements comprise the following points:

\begin{enumerate}
	\item \lcs aims to be an easy to use tool.
	\item None of its functionality should require the user to invest a great amount of work.
	\item It is preferred to hide as much complexity as possible from users.
	\item The web interface of the service is required to be fast, clean and responsive, since it will be amongst the first impressions users get of SCION.
	\item The web interface needs to be visually appealing, both on desktop and mobile devices.
	\item In terms of maintainability, \lcs aims for easy extensibility as the project evolves fast and new requirements need to be integrated with minimal effort.
\end{enumerate}