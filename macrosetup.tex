%% Custom commands
%% ===============

%% Special characters for number sets, e.g. real or complex numbers.
\newcommand{\C}{\mathbb{C}}
\newcommand{\K}{\mathbb{K}}
\newcommand{\N}{\mathbb{N}}
\newcommand{\Q}{\mathbb{Q}}
\newcommand{\R}{\mathbb{R}}
\newcommand{\Z}{\mathbb{Z}}
\newcommand{\X}{\mathbb{X}}

%% Fixed/scaling delimiter examples (see mathtools documentation)
\DeclarePairedDelimiter\abs{\lvert}{\rvert}
\DeclarePairedDelimiter\norm{\lVert}{\rVert}

%% Use the alternative epsilon per default and define the old one as \oldepsilon
\let\oldepsilon\epsilon
\renewcommand{\epsilon}{\ensuremath\varepsilon}

%% Also set the alternate phi as default.
\let\oldphi\phi
\renewcommand{\phi}{\ensuremath{\varphi}}

%% common abbreviations
\newcommand{\etal}{{et~al}.\@~}
\newcommand{\eg}{e.g.,\xspace}
\newcommand{\ie}{i.e.,\xspace}
\newcommand{\etc}{etc.\xspace}

% Paper-specific Macros (psms)
\newcommand{\name}{\textsc{Mondrian}\xspace}
\newcommand{\tp}{TP\xspace}
\newcommand{\tps}{TPs\xspace}

\newcommand{\as}{AS\xspace}
\newcommand{\ases}{ASes\xspace}
\newcommand{\asn}{ASN\xspace}
\newcommand{\asns}{ASNs\xspace}

\newcommand{\fnurl}[2]{\href{#2}{#1}\footnote{\url{#2}}\xspace}
\newcommand{\fnote}[2]{#1\footnote{#2}}\xspace
\newcommand{\fnoteurl}[3]{\href{#2}{#1}\footnote{{#3:} \url{#2}}\xspace}
\definecolor{code-gray}{gray}{0.92}
\newcommand{\code}[1]{\colorbox{code-gray}{\texttt{#1}}\xspace}
\newif\ifcomment
\newcommand{\cmnt}[1]{\ifcomment {\color{orange}{#1}} \fi}

% make bib section two columns and reduce font size
\renewcommand{\bibpreamble}{\begin{multicols}{2}}
\renewcommand{\bibpostamble}{\end{multicols}}
\renewcommand{\bibfont}{\footnotesize}

\newcommand{\customtoday}{\ifcase \month \or January \or February \or March \or %
	April \or May \or June \or July \or August \or September \or October \or November \or %
	December \fi \number \day, \number \year}


% magic setup to make multi-line footnotes align correctly
\makeatletter
\renewcommand\@makefntext[1]{\leftskip=2em\hskip-0.5em\@makefnmark#1}
\makeatother
