\begin{abstract}
 In order to increase SCION’s \cite{scion_book} availability and distribute it further, the SCIONLab project was created. SCIONLab is a publicly available version of SCION that allows research institutions and other interested parties to easily join the SCION testbed environment, thus making it possible to experiment with its unique capabilities. One goal of SCIONLab is to reduce the administration overhead for an institution to join and manage their own ASes. This management is done through the \lcs which serves two purposes. First, it offers an easy-to-use interface, enabling interested parties to register and download AS configurations to deploy onto their own hardware in order to become part of the SCION network. Moreover, \lcs is designed to be a global intermediary between the local management services of different ASes, managing connections between one another. The new features and components developed in this thesis aim to render the pre-existent Coordination Service much more robust and secure and extend its functionality to a point where it can be made available to the public. New features include the ability to send emails from \lcs, a mechanism to verify and activate new users, a counter measurement against bot abuse as well as new mechanisms to improve testing and deployment.
\end{abstract}