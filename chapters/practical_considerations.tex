\chapter{Practical Considerations}
\label{practical}

In this chapter, we discuss some practical considerations including functional and
management aspects, along with various deployment scenarios describing how \name can
be realized on today's enterprise infrastructure.


\section{NAT Devices}
\label{sec:nat}
Multiple hosts connected through a NAT device appear as a single host to the
\tp. \textit{Internal} NAT devices hardly pose a problem since the hosts under that NAT
device are all subject to the same subnet and therefore belong to a single network zone.
Network operators establish a security policy on the translated IP address, such
that \tps are able to authenticate the zone transit requests from/to a host under the
internal NAT device.

However, an \textit{external} NAT device located in an external network, e.g., carrier grade NAT,
could affect the \tp's secure tunneling ability. The translated \tp's IP address
would cause a MAC verification failure---recall that each symmetric key binds to the
triplet including \tps' IP addresses as described in \S\ref{sec:keymanagement}.
This, however, can be addressed by enforcing \tps to use their public IP addresses
to derive the symmetric keys. The keys are still secure since the first-level key
from which the pairwise keys are derived is exchanged with the CA-certified public
keys. Discovering the translated \tp address is also not a problem thanks to
the controller informing senders about the recipient \tp's address (see Protocol 3 in
Figure~\ref{fig:protocol}).

A potential operational failure occurs where multiple \tps reside behind the same NAT
device. This would lead to remote \tps deriving identical keys for all the \tps
behind the NAT. One possible solution would be to use a unique \tp identifier
instead. Since all such \tps would be under one administration domain, assigning
unique \tp identifiers upon bootstrapping is feasible. Then, the \tps convey
their identifiers in the AT header field alongside the destination zone ID. This
might slightly increase the size of the header, but does not degrade the
security of the underlying authentication.


\section{Tunneling Granularity} %\paragraph{Zone-to-Zone VPN}
\label{sec:granularity}
% Thanks to the flexible and scalable key derivation scheme introduced in~\cite{rot2020piskes},
% the key establishment supports the secure tunneling in different granularities: 
% i) site-to-site tunneling, ii) zone-to-zone tunneling, and iii) site-to-zone tunneling 
% (the one we take). 
Secure tunneling can be realized in different granularities: i) site-to-site tunneling,
ii) zone-to-zone tunneling, and iii) site-to-zone tunneling (the one used by \name).

% Similar to IPSec VPN, the site-to-site tunneling requires a single symmetric key per
% a \tp pair, e.g., the first-level key. Indeed, the site-to-site key establishment is 
% already enough in terms of communication security and privacy for two tunnel endpoints. 
% However, this approach requires another step of zone transfer verification.
Similar to IPSec VPN, site-to-site tunneling provides strong guarantees about
communication security and privacy for two tunnel endpoints. However, from a
flexibility and manageability standpoint, having a site-to-site tunneling
architecture is not ideal. Every tunnel endpoint needs to share a key with every
other endpoint with which it wishes to exchange data. This adds state to the
endpoints that needs to be kept in sync. Adding a new site requires an update on
all the other sites that wish to communicate with the new site. Then, yet
another layer of security middleboxes (e.g., firewalls) are required to perform
zone transfer authentication since keys do not designate a specific zone.

An alternative way of providing authentication is to use one key per zone pair.
In this model, when a zone transfer is required, a \tp would sign the data on
behalf of the zones with the corresponding source/destination zone key pair.
This approach has the benefit that sender and receiver get decoupled as in
principle any site that possesses the right keys can perform that zone transfer.
Adding a new site would be as easy as fetching the right keys for the desired
zone combinations. This process is independent of all the other sites. However,
a receiver \tp needs to be able to fetch the right keys from the cache, in order
to derive second-level keys, which means that the zone transfer information must
be visible in plain text. Thus, an attacker could potentially learn the zone
structure of the observed network. Also, having a separate key per zone pair
does not scale since the number of zones gets big very quickly for large
networks.

Driven by these considerations, we designed the new concept of site-to-zone
tunneling, which represents a middle ground combining the advantages of the two
approaches, the notion of secure tunneling and zone transfer authentication. The
symmetric keys are distinguishable depending on the destination zone, while at
the same time the zone-to-zone security policies are not being exposed. Thanks
to the flexible and scalable key derivation scheme introduced
in~\cite{rot2020piskes}, the key establishment does not expand state, while
still providing unique symmetric keys per zone. Furthermore, this scheme scales
linearly with the number of sites, not the number of zones, which ensures
scalability.

\section{Distributed Controllers}
\label{sec:distributedcontroller}
Driven by the scalability and reliability issues inherent to a single physical
controller, namely \textit{single-point-of-failure}, logically centralized
control-planes built on physically distributed instances find wide acceptance in
the current practice. The most common approaches realizing distributed
controllers can be broadly categorized into horizontal
distribution~\cite{berde2014onos,medved2014opendaylight} and hierarchical
distribution~\cite{hassas2012kandoo,yap2017taking}. Independent of which
distribution architecture is used, we discuss location, coordination, and
migration aspects of distributed controllers.

\paragraph{Location}
The notion of a logically centralized control-plane offers flexibility in
network design and management. A key design choice is placement of the
(distributed) controllers, which could impact to performance, reliability, and
management scalability of a given network. There is comprehensive research on
the controller placement problem considering practical issues from control
latency to reliability, from cost-optimization to load balancing, and so
on~\cite{das2019survey,zhang2017role,he2019toward}. Among those, we are mainly
interested in the latency performance indicator; that is the latency between a
controller and regional forwarding devices.

The best latency is achieved when each branch site has its own controller. By a
placement near local \tps, the controller minimizes the \tp-controller latency
for the zone transfer authorization protocol, allowing instant feedback for
packet forwarding---we note that inter-controller communication for global
coordination is commonly not latency sensitive. For the sake of control-plane
security, the controller resides in a highly restricted zone to which only the
local \tps and remote controllers have access. Although the per-site controller
offers the best performance regarding policy enforcement for the data-plane,
there might be a cost-efficiency problem for a large-scale network with
thousands of branches.

Alternatively, we consider a sparse distribution model, e.g., on edge-cloud
systems. Similar to today's cloud services, network operators running
geographically distributed data centers can instantiate multiple controllers at
the central point of regional branches. The control-plane latency overhead would
be higher compared to the dense deployment model---if the data center edges are
geographically diverse, the overhead could be minimized---but, in terms of
cost-optimization and management scalability, it could be a more viable
approach.

\paragraph{Coordination}
It is important to keep consistency in global coordination across the
distributed controllers. Indeed, inconsistency in security policy might grant
hosts with a low security clearance unauthorized access to highly restricted
zones, resulting in unexpected information leakage and eventually administration
failure. With this in mind, we consider a consensus algorithm with strong
consistency guarantees~\cite{panda2013cap,phemius2014disco,shi2014giraffe},
where the security policy is dynamically shared/replicated across the
distributed controller instances, ensuring concurrent policy enforcement towards
the data-plane devices. There are numerous open-source projects, such as
\fnurl{Consul}{https://github.com/hashicorp/consul}, \fnurl{Apache
	ZooKeeper}{https://zookeeper.apache.org/}, and
\fnurl{ETCD}{https://github.com/etcd-io/etcd} available.

\paragraph{\tp Migration}
To benefit from the distributed controller environment, a dynamic controller discovery
process also becomes important. That is, \tps should be able to search a cluster of best
candidates, diagnose the performances of control latency, and seamlessly migrate to the
best controller. To this end, we consider a two-step migration process: i) \tp-driven
control channel initialization and ii) controller-driven \tp migration.

\tps are responsible for establishing the first control plane channel with a controller.
For example, a new \tp ($\tp_{new}$) has been configured to contact an initial controller
acting as a first rendezvous point.
% \claude{a tp has been configured with an initial controller address acting as rendez-vous point?}
The initial information contains the controller's IP address
($C$), the corresponding zone ID ($Z_{C}$), and the \tp's IP address behind which the controller
resides ($\tp_{C}$). If the controller is located in a remote site (i.e., $\tp_{new}
	\neq \tp_{C}$), $\tp_{new}$ should connect with $C$ through $\tp_{C}$. Otherwise, e.g.,
$C$ is within the same LAN or in public network, $\tp_{new}$ can directly send $C$ a request
for control-plane channel establishment.

Once the \tp joined the network, the controller then initiates a migration process to
find the best controller ($C_{best}$) for $\tp_{new}$. Upon a migration request broadcasted
by $C$, other controllers measure the possible latency to $\tp_{new}$ and reply back
the results. Then, $C$ elects $C_{best}$ considering the latency measurements and the
current load balance, and sends $\tp_{new}$ a \texttt{RoleChange()} request containing
$C_{best}$, $Z_{C_{best}}$, and $\tp_{C_{best}}$. Finally, $\tp_{new}$ swaps the best
controller by establishing a new channel with $C_{best}$. The migration process
is also applied when changes in the network are detected.


\section{Nonce Reset}
\label{sec:nonce}

\begin{table}[htb]
	\footnotesize
	\caption{Comparsion of different nonce creation strategies. \texttt{l} is the length of the
		nonce, \texttt{s} $\leq$ \texttt{l} is the subset of bits reserved for the random part of the nonce,
		\texttt{p} is the number of nonces generated and \texttt{r} is the number of unplanned nonce
		resets.}
	\label{tab:nonce}
	\renewcommand\arraystretch{2}
	\begin{tabularx}{1.1\linewidth}{|X|X|X|X|X|X|}
		\hline
		                                        & \textbf{requires randomness}                       & \textbf{requires non-volatile memory} & \textbf{overlap probability w/o
		resets}                                 & \textbf{overlap probability for \texttt{r} resets} & \textbf{overlap}                                                                                                    \\
		\hline
		\textbf{counter only}                   & no                                                 & no                                    & 0                               & 1                                  & full \\
		\hline
		\textbf{randomness only}                & yes                                                & no                                    &
		$1-\frac{(2^l)!}{2^{l*p}(2^l-p)!}$
		                                        & $1-\frac{(2^l)!}{2^{l*p}(2^l-p)!}$                 & partial                                                                                                             \\
		\hline
		\textbf{counter paired with randomness} & yes                                                & no                                    & 0                               & $1-\frac{(2^s)!}{2^{s*r}(2^s-r)!}$
		                                        & full                                                                                                                                                                     \\
		\hline
		\textbf{reset points}                   & no                                                 & yes                                   & 0                               & 0                                  & -    \\
		\hline
	\end{tabularx}%
\end{table}

% Authenticators are created using an AEAD algorithm with an underlying AES-128 cipher in GCM mode.
% GCM requires a nonce as input argument to build counters which are then used to create the
% keystream. 
The same nonce must never be used twice with the same key, otherwise the security of the
cipher significantly decreases. In theory it is easy to create nonces that fulfill this
requirement. One can simply use a counter which is increased for every invocation of the
AEAD algorithm. In real systems this is not so easy to achieve since machines can crash
and lose their state, specifically their nonce counter. Outlined below are some techniques
to approach this problem.

\begin{itemize}
	\item \textit{Purely random nonce}: uses all bits of the nonce for randomness. Low
	      probability for overlap but no guarantee even when no resets.
	\item \textit{Counter paired with random sequence}: divides the nonce into a counter
	      and a randomized part. Initialize the random part after every restart and increase
	      the counter part for every packet. On reset start counter from zero with a fresh
	      randomized part. In case of overlap all packets do overlap.
	\item \textit{Reset points}: defines specific resets points for the counter which are
	      stored on non-volatile memory (NV-memory). Increment counter in memory and write the
	      next reset point to NV-memory when threshold is crossed. On crash restart counter from
	      reset point on NV-memory.
\end{itemize}

Table~\ref{tab:nonce} shows a comparison of the mentioned approaches. For this work, we chose the
approach of pairing an 8 byte counter with 4 bytes of randomness ($\texttt{s} = 32$) to obtain a 12
byte nonce. An example calculation shows that for $\texttt{r} = 10'000$ resets, the probability of
a nonce clash is still only $\sim$1.16\%.

\cmnt{uncomment DMZ section}
% \subsection{DMZ Traffic}
% So far, we mainly described on how two \tps can securely communicate each other. However,
% since most of cooperative information systems have public services facing the public Internet,
% we also need to consider 

% DMZ is a zone hosting public services. Since access traffic to the public services is 
% mostly originated from public Internet, 

\section{Fast Failover}
\label{sec:ecmp}
Operating multiple \tps with advanced ECMP-enabled layer 2 protocols (e.g., SPB and
TRILL) is a viable network design that provides load balancing and enhanced resiliency
against a \tp or link failure, as we discussed in \S\ref{sec:disrupting}.
If a \tp is unable to continue data transmission, ECMP engages. It searches an alternative
forwarding path considering cost equality and redirects flows through the new path, assuring
continuous communication for end hosts~\cite{rfc6754}.
This elastic migration during a transient network outage, however, does not replicate the
forwarding state of each flow, and thus might require another step of repopulating
forwarding state. Upon packet arrival, the new \tp fetches the zone transfer policy
and caches it in the forwarding table. Note that for \texttt{established} connections this might
cause unexpected packet drops if the first packet is a response. Nevertheless, the
communication would continue as soon as the original sender notices the packet drop
and requests a retransmission.

\cmnt{uncomment cross-enterprise section}
% \subsection{Cross-enterprise Setup}
% \label{ssec:crossconnection}
% Collaborators may join to an enterprise network. To be able to access highly secure zones
% in the enterprise network, the collaborators need to satisfy the following administrational
% and technical constraints: i) the collaborators must acquire a proper access grant from
% the enterprise's administration, and ii) install a \tp at their network.

% The enterprise network would have a single administration domain as we assumed
% in~\S\ref{ssec:assumptions}. To allow the collaborators interconnect their network system

% \subsection{\tp Deployment}
% \label{ssec:deployment}

% From a technical perspective, a \tp has obligations of: i) interconnecting VLANs , ii) 
% fetching security policies from the controller, and iii) enforcing the policies. 


% \paragraph{Gateway}
% A gateway 


% \paragraph{Middlebox}



\section{Incremental Deployability}
\label{sec:deployability}

To be incrementally deployable, \name does not require changes from end hosts nor the local network
infrastructure. \name can take a supportive role by complementing already installed lines of defense
such as firewalls, IPS, and IDS. For instance, traffic can be pre-filtered by \tps before it reaches
firewalls located deeper inside the network. This approach favors an incremental deployment
strategy. On the other hand, \name can also be used as a single, all-in-one solution providing
packet filtering, tunneling, and routing within one device. Such a deployment is especially
interesting for small branch sites with much simpler network layouts. Here, \name can drastically
reduce the number of devices that need to be maintained.



