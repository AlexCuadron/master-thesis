\chapter{Conclusion}
\label{concl}

\section{Summary}
\label{ssummary}

Network zoning has long been recognized as the cornerstone of secure network operation and management.
In the current practice, operators realize network zones with network segmentation technologies and security middleboxes.  
As information systems become more dynamic from a topological, operational, and functional perspective, however, the conventional network-zoning architectures face new challenges in terms of scalability and flexibility.
In this paper, we have shown that lightweight policy enforcement for inter-zone communication
is achievable. 
Following a constructive approach with a cryptographic foundation, it is possible to create a proactive alternative to the mostly reactive systems presently used in network zoning.
In conjunction with \name, verification based on firewalls becomes simpler because firewalls would only process a limited amount of (filtered) traffic.
\name consequently reduces the number of management points of distributed networks while retaining a high degree of security.
% % \claude{\sout{In this paper, we introduce \name, a new network zoning architecture that enables secure, scalable, and lightweight security enforcement for large-scale corporative networks. s
% % The new concept of a global transit zone simplifies the hierarchical zone structure, transforming it into a flat and orthogonal structure in order to provide high flexibility and security for zone design.}}
% \claude{In this work we have shown that lightweight policy enforcement for inter zone communication is achievable by introducing a globally distributed transit zone.}
% \claude{Following a constructive approach based on a cryptographically sound foundation we have shown that it is possible to create a proactive alternative to the mostly reactive systems presently used in network zoning.} 
% % \claude{\sout{By ensuring that only cryptographically authorized packets can transit through the unified Zone Translation Points, \name ensures lightweight policy enforcement in inter-zone communication.}}
% % \claude{\sout{Furthermore, no changes are required for endhosts and local networks, achieving compatibility with the existing information systems and thus providing deployability.}} 
% \claude{This does not require endhosts and local network infrastructures to change, thus achieving compatibility with existing information systems and therefore deployability.}
% \claude{A system like \name allows an enterprise to reduce the number of management points of its distributed network while maintaining a similar degree of security.}
% % We believe that the new notions of global transit zone and unified zone-translation 


\section{Future Work}
\label{sfuture}

Focus on privacy
\begin{itemize}
	\item hide zone id
	\item pad to full message length
	\item always send traffic (jondo networks)
\end{itemize}
 