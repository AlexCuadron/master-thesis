\chapter{Conclusion}
\label{concl}

In this work, we present the first comprehensive empirical study of Large Reasoning Models (LRMs) in agentic environments. We identify a fundamental challenge: the \textbf{Reasoning-Action Dilemma}, in which models must balance environmental engagement against internal reasoning about potential actions. Our analysis reveals that LRMs consistently favor internal simulation over environmental interaction, a behavior we define as \textbf{overthinking}.

Through our systematic evaluation framework, we analyzed 3,908 trajectories using a novel overthinking score metric. Our findings demonstrate a strong correlation between overthinking and task failure rates, with reasoning models showing particularly high vulnerability to this phenomenon compared to their non-reasoning counterparts.

Our research demonstrates that even simple interventions to mitigate overthinking can yield substantial benefits: a 43\% reduction in inference costs while improving resolution rates by 25\%. These results, combined with our observations about the effectiveness of function-calling capabilities and targeted reinforcement learning, suggest promising directions for developing more efficient and environmentally grounded reasoning models.
