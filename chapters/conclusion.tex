\chapter{Conclusion}
\label{concl}

\section{Summary}
\label{ssummary}

Network zoning has long been recognized as the cornerstone of secure network
operation and management. In the current practice, operators realize network
zones with network segmentation technologies and security middleboxes. As
information systems become more dynamic from a topological, operational, and
functional perspective, however, the conventional network-zoning architectures
face new challenges in terms of scalability and flexibility. In this thesis, we
have shown that lightweight policy enforcement for inter-zone communication is
achievable. Following a constructive approach with a cryptographic foundation,
it is possible to create a proactive alternative to the mostly reactive systems
presently used in network zoning. In conjunction with \name, verification based
on firewalls becomes simpler because firewalls would only process a limited
amount of (filtered) traffic. \name consequently reduces the number of
management points of distributed networks while retaining a high degree of
security.

\section{Future Work}
\label{sfuture}

The work presented in this thesis can be extended along different axes. For
one, it would be interesting to explore how an architecture like \name can be
used to provide privacy for inter-domain data transmissions. \name provides
confidentiality and integrity on the data in transit. However, certain traffic
patterns and packet sizes could still allow an attacker to draw conclusions about
what sort of traffic is being sent (\eg sensor data, control messages, file
transmissions) and which security zones are located
at a given site. If privacy is a concern, a first step could be to extend \tps
with new modules that pad all packets to the full MTU length and insert mock
traffic, such that traffic patterns are hidden.

Another direction in which this work could be extended is the implementation
of the \tp gateway in a high-performance language. Various operations---in
particular cryptographic ones---would benefit from an optimized
implementation on dedicated network hardware.

Lastly, it would be interesting to explore how certain properties of local
networks can be conveyed across the WAN to a remote network. For example, technologies such as
SPB and TRILL have support for layer 2 multipathing, the corresponding control
messages are, however, not transmitted by \name. A deeper interoperability of
different protocols across network boundaries would certainly be a desirable property.
