\chapter{Security Analysis}
\label{analysis}

% Security analysis goes here.

% Compromise attack: sepration of key management from controller limites impact of 
% compromised controller. 

% Source spoofing from internal host -> internal network is assumed to be trusted.
% -> internal attacker does not know which IP address (or subnet) is allowed to
% access a target zone.
% -> port-based ingress filtering (enterprise network benefits from the deployment)

% We analyze the security properties that \name provides, considering the threat model
% introduced in \S\ref{ssec:threatmodel}. The attack classes described here are threefold:
% i) attacks to identify a target network structure in terms of security policies and their
% corresponding access hierarchies, ii) attacks to infiltrate restricted and highly secure
% zones without a proper permission, and iii) disruption attacks that destroy network 
% functionalities. For each attack class, we describe potential attack methodologies and
% mitigations.
We analyze the security properties that \name provides, considering the threat model
introduced in \S\ref{sec:threatmodel}. The attack classes described here are twofold:
i) attacks to infiltrate restricted and highly secure zones without proper permission,
and ii) disruption attacks that prevent availability.
% For each attack class, we describe potential attack methodologies and mitigations.

% \subsection{Zone Structure Surveillance}
% \label{ssec:scanning}

% % In this attack class, we consider attackers sniffing the victim network to disclose
% % security vulnerabilities. Attackers often take a long breath to successfully breach 
% % the highly secure systems. By either passive monitoring or active scanning,
% % attackers attempt to draw a map of the target network and identify the topological
% % characteristics along the access controls. Note that, since this attack class is 
% % usually in preparation for major security breaches, we only consider attackers from 
% % offshore network.

% % \paragraph{Passive Monitoring}
% % Attackers may observe packets at the traffic aggregation points, e.g., routers and switches
% % at core networks. Since there are countless packet-sniffing tools and wire-tapping devices 
% % available in the market, indeed monitoring live traffic associated with the target network
% % is an effortless---just time consuming---and effective attack method; the header fields is 
% % already informative to the attackers because they expose the pairs of communicating hosts, 
% % enabling attackers correlating the hosts and their trust hierarchy. 

% In this attack class, we consider attackers sniffing the victim network to disclose
% security vulnerabilities. Note that, since this attack class is usually in preparation 
% for major security breaches, we only consider attackers from offshore networks.

% \paragraph{Passive Monitoring}
% Attackers often take a long breath to successfully breach the highly secure systems.
% Monitoring live packets at the traffic aggregation points (e.g., router and switches at the 
% core network) is definitely time consuming, but indeed an effective and effortless attack
% method; the packet header is already informative for attackers, as it exposes the pairs
% of communicating hosts, enabling attackers to correlate the observed hosts and their access
% hierarchy (i.e., zone map). In addition, there are countless packet-sniffing tools and 
% wire-tapping devices available on the market. 

% Nonetheless, \name's secure tunneling for the inter-domain zone transfer prevents passive 
% monitoring from sketching the zone map. It only exposes the IP addresses of the two tunnel endpoints,
% while keeping data confidentiality and communication privacy by encrypting the original IP 
% packets. Although a \name packet conveys the destination zone ID in the AT header field, it
% only depicts what zones are under the destination \tp. Furthermore, the zone ID is an
% arbitrary number, and therefore it is impossible to infer the underlying security policies.

% \paragraph{Active Scanning}
% Attackers may perform active scanning to identify a vulnerable spot by transmitting forged
% packets to a specific host residing in a trusted zone. To this end, the attackers however
% must pass through the authentication check which requires a correct cryptographic key.
% Since the key binds to the triplet of \{$\tp_{src} | \tp_{dst} | zoneID_{dst}$\} and only
% the corresponding \tps can derive the correct key, scanning the network from outside
% is hardly possible. 


\section{Infiltration without Permission}
\label{sec:infiltration}

One of the main attack objectives is to access valuable information assets
protected by access constraints.

\paragraph{Man-In-The-Middle Attack}
To become ``in the middle'', an attacker could initiate independent communication channels
with two \tps and relay messages between them. By using the attacker's public key for the
channel establishments, the attacker is able to generate valid packets that can bypass the
\tp's authentication check.

\name's PKI design prevents MITM attacks. A MITM attack
can succeed only if the attacker convinces each \tp that they are talking to each other.
In the process of secure channel establishment, however, \tps authenticate each other using
the certificates issued by the mutually trusted CA (e.g., enterprise). Since the attacker's
public key cannot be certified by a valid certificate issued by that CA, a MITM attack will fail.


\paragraph{Packet Replay}
Attackers can observe valid \name packets and then reuse them to transmit attack traffic.
Nevertheless, the validity of the reused packet header will be compromised once the payload
is changed---recall that the authentication scope covers the entire packet including
AT and EIP as shown in Figure~\ref{fig:header}---and therefore attackers cannot
successfully pass the authentication check at the recipient \tp.


\paragraph{Brute-force Attack}
Another approach is to brute-force the key used for authentication or the MAC. As we are using 128-bit cryptographic keys and MACs, such attacks are currently infeasible. To achieve resilience to quantum computers, the key size would need to be doubled, however.

% Attackers may
% generate all possible combinations to derive the valid key. To break the 128-bit symmetric
% key which has $2^{128} \approx {3.4e38}$ combinations, it requires $2.59e13$ years on the
% world's fastest supercomputer, Fugaku ($415.53e15$ Flops)~\cite{top2020supercomputer}, even 
% if we consider that a combination check requires just a single Flop.

% To brute-force the 128-bit MAC instead, it requires ${3.4e38}$ probes to be 
% transmitted, taking $6.38e22$ years on a {100} {Gbps} network link by using the smallest
% frame size of 74 bytes (i.e., Ethernet(18 bytes) + IP(20 bytes) + AT(36 bytes)).

\section{Denial-of-Service Attack}
\label{sec:disrupting}

\paragraph{Exhaustion Attack}
Flooding the \tp's authentication process could be an effective attack vector. Attackers
may attempt to forward a large number of packets to the target \tp in order to exhaust the
\tp's resources. Even if the attack packets contain invalid authentication tokens, the \tp still
needs to verify these tokens which wastes resources and prevents legitimate packets from getting through.

To be resilient to such attacks, we consider operating multiple \tps at the entry points
of cooperative networks. Multiple \tps enables network operators
to load balance and easily switch over to another \tp in case of a link (or a \tp)
failure. In fact, many data centers and large enterprise networks already employ equal-cost
multipathing (ECMP)~\cite{rfc2991,rfc2992} along with multiple gateways and ToRs (Top-of-Rack
switches) to provide reliable intra-networking services. In addition, to mitigate possible DoS
attacks from the public Internet, Microsoft implemented a global ECMP
infrastructure~\cite{ms2020ecmp}. Path-aware networking along with multipath communication
% is also in the spotlights of both, industry and academia recently
also enables active switching to different entry points, if some fail or are under DDoS attack.~\cite{Dawkins2018,Trammell2018}.


% \paragraph{Fake \name entities}
% Another possible attack factor is to initialize a face \name entity. 
