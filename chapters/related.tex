\chapter{Related Work}
\label{related}

% \paragraph{Security Zoning}
% During stacking our knowledge of security zoning, we noticed that there is a limited number of academic research on network zoning architecture while considerable research is underway in the industry regarding access management and system design. 
% firewall architecture 
% 1. standalone
% 2. distributed
% 3. virtualized
% 4. colaborative 
The majority of literature in network zoning has focused on security enforcement
architecture using middleboxes such as firewalls, IPS, and IDS. Conventional
security middleboxes define restricted zones and filter unwanted traffic at the
entry points of protected zones~\cite{cheswick1994firewalls}. As information
systems and the corresponding network functions got more complicated, the notion
of distributed security systems has been introduced in the late
1990's~\cite{bellovin1999distributed}. Early approaches to protect only internal
information systems from external threats have further evolved to mitigate
sophisticated threats, for example insider attacks, rule tampering,
application-level proxies, and denial-of-service
attacks~\cite{markham2001security}. Later, with emerging network virtualization
technologies and cloud computing environments, virtual firewalls and
collaborative security enforcement kept getting attention from both academia and
the industry~\cite{liu2008collaborative,yu2017psi}.

Despite numerous research efforts, network zoning using security middleboxes has
issues with respect to performance---the iMIX throughput degrades by 40 - 75\%
on commodity products~\cite{juniper2020comparison}---and
misconfigurations~\cite{fayaz2016buzz,tschaen2016sfc,yuan2020netsmc}. With
\name, we address these challenges by leveraging cryptographic policy
enforcement and centralized policy orchestration, achieving scalable, effective,
and cost-efficient network zoning.

% Zoning is orthogonal with all the concepts. Zoning is a fundamental piece.
% Zoning need to be done first in scale passion to decide which node can
% communicate which other nodes, then put firewalls, ids, and ips. All the the
% previous technologies is for single zone. Given the complexity, 

% \paragraph{Network Segmentation and Isolation}
Network isolation through network segmentation is another essential element of
network zoning. To logically segment the physical network, network
virtualization technologies are heavily used in today's Internet, in particular
large enterprise networks and cloud computing environments.
VLAN~\cite{ieee2018vlan} is the most frequently used network segmentation
technique. It logically segments a physical LAN into up to 4094 virtual LANs by
tagging the layer 2 header with a unique VLAN identifier (VID). Later, Virtual
eXtensible LAN (VXLAN)~\cite{rfc7348} has been introduced for better
scalability; it expands the number of virtual LANs to up to 16 million by
leveraging a 24-bit identifier. SPB~\cite{ieee2012spb} and
Trill~\cite{rfc6325,rfc7176} are layer 2 routing protocols enabling multi-path
communication among virtual LANs within the same physical LAN. Although security
is an important property when it comes to network segmentation, it unfortunately
has often not been treated as a major concern. Systems lack protection for
segment membership and access control. To isolate each segment, use of a large
number of security middleboxes is necessary, thus increasing operational costs
and management complexity.

SVLAN~\cite{kwon2020svlan} is an architecture enhancing security in network
isolation by enforcing the receiver's consent towards incoming traffic. SVLAN is
the closest work related to \name, but there are three major differences. First,
\name decouples policy establishment and enforcement. In SVLAN, the
authorization delegate (i.e., controller) is responsible for establishing basic
reachability policies and issuing authorization tokens to enforce the policies.
The single trust model requires additional latency to fetch the authorization
token for a new flow. In contrast, \name leaves the policy enforcement to the
data-plane, reducing overhead. Second, \name provides data confidentiality.
SVLAN focuses on bridging layer 2 networks and thus leaves higher-layer
protocols responsible for communication security. Lastly, \name is compatible
with various Internet architectures, whereas SVLAN is relying on architectures
that make use of segment
routing concepts~\cite{rfc8402,rfc8660,Perrig2017}. Although segment routing is an
emerging technology, it is not yet completely supported, especially for
collaborative wide-area networks, and thus constrains deployability.

