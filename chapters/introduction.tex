\chapter{Introduction}
\label{intro}

% about network security zoning
Network zoning has long been an essential part of the Internet security infrastructure,
which logically partitions network and information assets into disjoint segments that share the
same security requirements and policies, and functional similarities. Zones define the network boundaries
and their defense
requirements by stating the entities populating the zones, the entry points into the zones, and
how traffic is monitored and filtered at these entry points. Informally, these zones
are realized by a virtualized separation at layer 2 (e.g., IEEE 802.1q~\cite{ieee2018vlan})
with firewalls at higher levels governing data transfers between zones~\cite{mayer2000fang}.

Each zone is identified by a distinct level of trust, and
forms a trusted/untrusted relationship with other zones~\cite{obregon2015infrastructure}.
To realize the unidirectional trust model, firewalls are considered to be the most viable
technology and are widely used in the current practice. However, operating firewalls in
large enterprises is often challenging for network operators and security architects. The
access control for network zones might be dynamic, and thus it requires complex
management schemes to accommodate a myriad of policies. While there are advanced
technologies such as virtual firewalls~\cite{deng2015vnguard,bakker2016network}, distributed
security enforcement~\cite{markham2001security,yu2017psi}, and Unified
Threat Management (UTM)~\cite{qi2007towards}, newly designed to enforce access control polices in extremely
dynamic networks, network zone management and modeling
still remains cumbersome~\cite{ramasamy2011towards,gontarczyk2015blueprint}.

% limitations on the current zoning techonologies
% > secure communication for inter-domain zones (vxlan, ipsec)
% > cost inefficiency (expensive lease line and firewalls)
% > management scalability (key management)
Bridging geographically distant network zones is very challenging today. In general,
network zones are created not only for security purposes but also because of geographical,
operational, or organizational factors. Large enterprises with geographically distributed
branch networks, and possibly collaborative partners' networks need to be interconnected.
Given that distant network zones exchange information over an untrusted
network (e.g., the Internet), there is a risk that the communication exposes security-sensitive
information during transit. To mitigate
such threats, administrators leverage additional security mechanisms (e.g.,
IPsec~\cite{rfc4301} and SSL-VPN~\cite{sun2011advantages})
which ensure confidentiality
and integrity of the transmission over the untrusted network by encrypting and authenticating the data with
securely shared cryptographic keys. Nonetheless, these technologies bring forth new challenges
such as management scalability~\cite{felsch2018dangers} and compatibility issues with other
security solutions~\cite{liu2008collaborative}---universal agreement with business partners on building collaborative security infrastructure is often problematic.

% the notion of distributed data centers

% TP summary
% > secure zone transfer over wan
% > simplified zoning architecture (logical transit zone)
% > cost efficient: firewall x, lease line x
% > easy key management (management scalability)
\name is a new network zoning architecture that secures inter-zone communication---which
operates on layer 3, supporting heterogeneous layer 2 architectures---while ensuring
scalable cryptographic-key management and flexible security policy enforcement.
\name flattens the current hierarchically-complex network zone topology into a collection of
\cmnt{we never show what hierarchy was used before}
horizontal zones connected to a unified security gateway, called Zone Translation Point
(\tp), thus simplifying large enterprise networks.
By interconnecting zones through \tps, complex zone restructuring operations become
easier with respect to new zone initializations or zone migrations.
The \tp ensures source authentication, zone transfer authorization, and illegitimate access filtering by
acting as a secure ingress/egress point for network zones. A logically centralized control unit provides
management scalability on zone classification and policy enforcement, and mediates cryptographic key
establishment.

A secure zone transfer is performed in three steps: i) the security gateway acquires
access policies for each network zone from its controller, ii) the gateway issues a
cryptographically protected authorization token if a given zone transfer request is
permitted, and iii) the network forwards only packets with a valid token. By leveraging
the notion of secure tunneling between two endpoints (i.e., a pair of local and remote \tps),
confidentiality and integrity of the zone transfer packets are ensured, while keeping the overhead of
the authentication process small. For scalable key management,
we employ a key establishment system that enables dynamic key derivation
and ensures perfect forward secrecy.

% Evaluation
We provide an implementation of \name that ensures secure zone transfer for
both intra and inter-domain communication at line rate, while requiring no network-stack
changes from end hosts. We extensively evaluate this implementation to demonstrate
the practical viability of \name. The results show that the \tp introduces negligible
processing delay; less than \SI{500}{ns} of additional delay for intra-domain zone transfer
and approximately $2.5 \sim 3.5 \mu$s for inter-domain zone transfer traffic. We further
provide in-depth security and practicality analyses.

% contributions
The main contributions of this paper are the following:
\begin{itemize}
	\item We introduce \name, a new security architecture that enables secure, flexible
	      and viable network zoning and inter-zone communication for large enterprise
	      networks.
	\item We introduce the new notion of an inter-domain transit zone that dramatically
	      simplifies the current hierarchical zone structure, enabling flexible and cost-efficient
	      network zone management.
	\item \name enables network filtering at the edge of the network, such that it suppresses
	      network overhead in terms of wasted network bandwidth and packet processing time.
	\item We implement \name as an opensource project.
\end{itemize}

