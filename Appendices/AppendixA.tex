\chapter{Security and Maintainability Enhancements}
\label{misc}

\section{Ensuring Secure Operation}

The sections below outline less extensive but nonetheless critical enhancements made to ensure a secure and stable operation of \lcs.

\subsection{Server Crash on Login}

Attempting to log in to \lcs with either an empty email address or an empty password caused the back end server to crash, effectively taking \lcs offline. This was caused by malfunctioning error handling code, resulting in a nil pointer dereference. This defect exposed the service to accidental and malicious attacks, threatening its uninterrupted operation.

The security hole was fixed by correcting the handling of empty user information submitted to \lcs. 

\subsection{Nil Pointer When Accessing User Information}

For loading and displaying data corresponding to the right user on the web interface, \lcs depends on user sessions. The code responsible for validating these sessions contained malfunctioning error handling code which could have led to server crashes caused by nil pointer deferences.

To secure uninterrupted operation, the faulty code was repaired by correcting the specific error handling.

\section{Maintainability}

The following enhancements were made to increase \lcs's maintainability.  At the same time, they make sure the service operates flawlessly and retains its compatibility to SCION.

\subsection{Duplicate Configurations}

On startup, \lcs loads configurations from a configuration file and stores them in program variables. The \fnurl{goconf}{https://github.com/sec51/goconf} package used for this functionality allows configurations to be initialize with default values in case they are not set in the configuration file. This regularly caused confusion since certain settings were set in two different places, where changing it in the wrong place didn't have any effect.

This duplication was resolved by removing default values for configurations.

\subsection{Obsolete HTTP Handling Code}

Alongside code used in production, \lcs contained dead legacy code for serving web pages, including the web page resources themselves. This constantly led to confusion since changes made in the wrong place didn't have any effect on the system.

The code base was cleaned and restructured by removing unused code and resources.

\subsection{Obsolete Database Queries}

The package used to interface with the MYSQL database contained queries which were not needed any more.
For the sake of a cleaner code base, these queries were deleted.

\subsection{Outdated Dependencies}

\lcs uses a variety of libraries to offer its functionality. These libraries are downloaded and supplied by the \fnote{vendoring}{In the context of Go, vendoring denotes the process of copying the software packages a project relies on and storing them in the project repository.} software \fnurl{govendor}{https://github.com/kardianos/govendor}. Because packages were not updated regularly, \lcs was unaware of changes in the main \fnurl{SCION}{https://github.com/netsec-ethz/scion} code base it depends on. This led to incompatibility of SCION and \lcs.

In order to retain compatibility, govendor was instructed to vendor more recent packages. The code base was adapted to work with the changes introduced in SCION.

